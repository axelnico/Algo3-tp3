\section{Informe de modificaciones}

A continuación se señalan los cambios realizados para la reentrega del trabajo:

\begin{itemize}
	\item{
		Ejercicio 1
		\begin{itemize}
			\item Se realizaron correcciones de redacción en la sección de detalles implementativos.
			\item Se cambiaron las escalas de algunos gráficos a logarítimica
		\end{itemize}
	}
	\item{
		Ejercicio 2
		\begin{itemize}
			\item Se realizaron correcciones de redacción en la sección de detalles implementativos.			
			\item Se realizaron correcciones de redacción en la sección de comportamiento del algoritmo.
			\item Se dividió el pseudocódigo del algoritmo en partes, y se corrigió la explicación. 
			\item Se corrigió en la explicación de la complejidad, el análisis de las operaciones de las listas. 
			\item Se añadieron más experimentos, con más instancias y corrigiendo su explicación. 
			% valores para no tener picos al comienzo. Esto además se aclara en la
			% descripción de la sección.
		\end{itemize}
	}
	\item{
		Ejercicio 3
		\begin{itemize}
			% \item En la experimentación se reemplazó la cota de complejidad de $N^{N+2}$ por $N^3 \times N!$. Esto incluye la función utilizada como cota en los gráficos.
			% \item En la experimentación se reemplazó la constante de la cota teórica $0.001$ por $0.1$.
			% \item Se agregó pseudocódigo en la explicación de las podas para una mejor comprensión.
		\end{itemize}
	}
	\item{
		Ejercicio 5
		\begin{itemize}
			\item Se cambiaron los experimentos por 2 nuevos con instancias nuevas generadas de manera aleatorias
			\item Se agregaron conclusiones de la experimentación
	}
\end{itemize}
