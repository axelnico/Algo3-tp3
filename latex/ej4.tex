\section{Ejercicio 4: Metaheurística}

  % \begin{figure}[ht]
  %   \begin{center}
  %     \includegraphics[width=0.5\columnwidth]{imagenes/pacman.png}
  %     \caption{Perdidos y con poca fuerza}
  %   \end{center}
  % \end{figure}

    % 1. Describir detalladamente el problema a resolver dando ejemplos del mismo y sus soluciones.
    \subsection{Descripción del problema y solución propuesta}
        En este punto, se pidió realizar un algoritmo basado en una metaheurística que resuelva el problema en cuestión. El mismo al tratarse de una metaheurística puede no dar la solución óptima, y puede no devolver una solución aunque la misma exista.

    % 2. Explicar de forma clara, sencilla, estructurada y concisa, las ideas desarrolladas para la resolución del problema. Utilizar pseudocódigo y lenguaje coloquial (no código fuente). Justificar por qué el procedimiento resuelve efectivamente el problema.
        Se decidió utilizar como metaheurística para resolver esta variante del TSP, GRASP. La misma consiste en lo siguiente: se obtiene una solución (en caso de ser posible) mediante una heurística greedy. La misma es similar a la desarrollada en el ejercicio 2. La diferencia es que se le agrega aleatoridad, ya que no se toma a la estación más cercana sino que se eligen k estaciones más cercanas (siendo k un parámetro configurable) y se elige una al azar de esas estaciones. La misma estación elegida tiene que ser factible de ser visitada. En caso contrario, el algoritmo devuelve que no encontró solución. Si se toma como parámetro k = 1, este algoritmo greedy es exactamente igual al desarrollado en el ejercicio 2 ya que en todas las iteraciones va a elegir al más cercano. Si se toma como k la cantidad de estaciones total, el algoritmo se transforma en uno completamente aleatorio ya que en cada iteración se toma cualquiera estación. Luego de obtenerse una solución a la misma se le aplica la búsqueda local desarrollada en el ejercicio 3 para tratar de mejorarla. Este procedimiento se repite varias veces hasta que se cumple la condición de parada (parámetro configurable). En cada paso, se va guardando la solución mejor que se encuentra. Es decir, se mantiene una solución, que se actualiza en caso de que se encuentre una mejor en cada iteración de este procedimiento. Al finalizar se retorna la mejor solución encontrada.